\documentclass[titlepage,a4paper]{article}

\usepackage{a4wide}
\usepackage[colorlinks=true,linkcolor=black,urlcolor=blue,bookmarksopen=true]{hyperref}
\usepackage{bookmark}
\usepackage{fancyhdr}
\usepackage[spanish]{babel}
\usepackage[utf8]{inputenc}
\usepackage[T1]{fontenc}
\usepackage{graphicx}
\usepackage{float}
\usepackage[left=2.5cm, top=2.5cm, right=2.5cm, bottom=2.5cm]{geometry}

\usepackage{underscore} % Permite usar el carácter _ como literal

\pagestyle{fancy} % Encabezado y pie de página
\fancyhf{}
\fancyhead[L]{TP1 - File Transfer}
\fancyhead[R]{Redes- FIUBA}
\renewcommand{\headrulewidth}{0.4pt}
\fancyfoot[C]{\thepage}
\renewcommand{\footrulewidth}{0.4pt}

\begin{document}
\begin{titlepage} % Carátula
    \hfill\includegraphics[width=6cm]{img/logofiuba.jpg}
    \centering
    \vfill
    \Huge \textbf{Trabajo Práctico 2 —  Software-Defined Networks}
    \vskip2cm
    \Large [TA048] Redes \\
    Curso 2 \\ 
    \vfill
    \begin{tabular}{ | l | l | l |}
      \hline
      Alumno & Número de padrón & Email \\ \hline
      Lucas Oshiro & 107024 & loshiro@fi.uba.ar \\ \hline
      Martin Reimundo & 106716 & mreimundo@fi.uba.ar \\ \hline
      Franco Agustin Rodriguez & 108799 & frodriguez@fi.uba.ar \\ \hline
      Mateo Riat Sapulia & 106031 & mriat@fi.uba.ar \\ \hline
      Ignacio Vetrano & 106129 & ivetrano@fi.uba.ar \\ \hline
    \end{tabular}
    \vfill
    \vfill
\end{titlepage}

\tableofcontents % Índice general
\newpage

\section{Introducción}\label{sec:intro}

    ...

\section{Hipótesis y supuestos realizados}\label{sec:supuestos}

...

\section{Implementación}\label{sec:implementacion}

...

\section{Pruebas}\label{sec:pruebas}

...

\section{Preguntas}\label{sec:preguntasAResponder}

    \subsection{¿Cuál es la diferencia entre un Switch y un router? ¿Qué tienen en común?}

    ...
        
    \subsection{¿Cuál es la diferencia entre un Switch convencional y un Switch OpenFlow?}

    ...
     
    \subsection{¿Se pueden reemplazar todos los routers de la Internet por switches OpenFlow? Piense en el escenario inter-ASes para elaborar su respuesta.}
    ...
    
    \section{Dificultades encontradas}\label{sec:dificultadesEncontras}
    Algunas dificultades que surgieron en el avance del proyecto fueron:
    
    \begin{itemize}
        \item ...
    \end{itemize}
    
\section{Conclusion}\label{conclusion}
    ...
    
\end{document}